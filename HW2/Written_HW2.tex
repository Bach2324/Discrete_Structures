\documentclass[12pt, letterpaper]{article}
\usepackage[margin=1in]{geometry}
\usepackage{amsmath}
\usepackage{amssymb}
\usepackage{fancyhdr}
\usepackage{pgfplots}
\pgfplotsset{compat=1.16}

\author{sudoTech}
\title{Discrete Structure Homework 2 -- Written}

\pagestyle{fancy}

\fancyhf{}
\lhead{Discrete Structures | Homework 2}
\rhead
{
    Bach Nguyen | ID: 104263571
}

\begin{document}

\textbf{Q1}. Propositions | p = "A is a knight" , q = "B is a knight", r = "C is a knight" \\

\setlength\parindent{40pt}\textbf{A}: "I'm a knight or B is a knave." can be translated to $(p \lor \neg q)$

\setlength\parindent{40pt}\textbf{B}: "A is a knight and C is a knave." can be translated to $(p \land \neg r)$

\setlength\parindent{40pt}\textbf{C}: "Myself and B are different." can be translated to $(r \oplus q)$\\

{\large\centerline{Truth tables}}

\begin{displaymath}
    \begin{array}{|c|c|c|c|c|c|c|c|c|}
    % |c c|c| means that there are three columns in the table and
    % a vertical bar ’|’ will be printed on the left and right borders,
    % and between the second and the third columns.
    % The letter ’c’ means the value will be centered within the column,
    % letter ’l’, left-aligned, and ’r’, right-aligned.
    \hline
    p & q & r & \neg p & \neg q & \neg r & p \lor \neg q & p \land \neg r & r \oplus q\\ % Use & to separate the columns
    \hline % Put a horizontal line between the table header and the rest.
    \textcolor{blue}{T} & \textcolor{blue}{T} & \textcolor{blue}{T} & \textcolor{red}{F} & \textcolor{red}{F} & \textcolor{red}{F} & \textcolor{blue}{T} & \textcolor{red}{F} & \textcolor{red}{F}\\
    \hline
    \textcolor{blue}{T} & \textcolor{blue}{T} & \textcolor{red}{F} & \textcolor{red}{F} & \textcolor{red}{F} & \textcolor{blue}{T} & \textcolor{blue}{T} & \textcolor{blue}{T} & \textcolor{blue}{T}\\
    \hline
    \textcolor{blue}{T} & \textcolor{red}{F} & \textcolor{blue}{T} & \textcolor{red}{F} & \textcolor{blue}{T} & \textcolor{red}{F} & \textcolor{blue}{T} & \textcolor{red}{F} & \textcolor{blue}{T}\\
    \hline
    \textcolor{blue}{T} & \textcolor{red}{F} & \textcolor{red}{F} & \textcolor{red}{F} & \textcolor{blue}{T} & \textcolor{blue}{T} & \textcolor{blue}{T} & \textcolor{blue}{T} & \textcolor{red}{F}\\
    \hline
    \textcolor{red}{F} & \textcolor{blue}{T} & \textcolor{blue}{T} & \textcolor{blue}{T} & \textcolor{red}{F} & \textcolor{red}{F} & \textcolor{red}{F} & \textcolor{red}{F} & \textcolor{red}{F}\\
    \hline
    \textcolor{red}{F} & \textcolor{blue}{T} & \textcolor{red}{F} & \textcolor{blue}{T} & \textcolor{red}{F} & \textcolor{blue}{T} & \textcolor{red}{F} & \textcolor{red}{F} & \textcolor{blue}{T}\\
    \hline
    \textcolor{red}{F} & \textcolor{red}{F} & \textcolor{blue}{T} & \textcolor{blue}{T} & \textcolor{blue}{T} & \textcolor{red}{F} & \textcolor{blue}{T} & \textcolor{red}{F} & \textcolor{blue}{T}\\
    \hline
    \textcolor{red}{F} & \textcolor{red}{F} & \textcolor{red}{F} & \textcolor{blue}{T} & \textcolor{blue}{T} & \textcolor{blue}{T} & \textcolor{blue}{T} & \textcolor{red}{F} & \textcolor{red}{F}\\
    \hline
    \end{array}
\end{displaymath} \\

\begin{displaymath}
    \begin{array}{|c|c|c|c|}
    % |c c|c| means that there are three columns in the table and
    % a vertical bar ’|’ will be printed on the left and right borders,
    % and between the second and the third columns.
    % The letter ’c’ means the value will be centered within the column,
    % letter ’l’, left-aligned, and ’r’, right-aligned.
    \hline
    p \iff (p \lor \neg q) & q \iff (p \land \neg r) & r \iff (r \oplus q)\\ % Use & to separate the columns
    \hline % Put a horizontal line between the table header and the rest.
    \textcolor{blue}{T} & \textcolor{red}{F} & \textcolor{red}{F}\\
    \hline
    \textcolor{blue}{T} & \textcolor{blue}{T} & \textcolor{red}{F}\\
    \hline
    \textcolor{blue}{T} & \textcolor{blue}{T} & \textcolor{blue}{T}\\
    \hline
    \textcolor{blue}{T} & \textcolor{red}{F} & \textcolor{blue}{T}\\
    \hline
    \textcolor{blue}{T} & \textcolor{red}{F} & \textcolor{red}{F}\\
    \hline
    \textcolor{blue}{T} & \textcolor{red}{F} & \textcolor{red}{F}\\
    \hline
    \textcolor{red}{F} & \textcolor{blue}{T} & \textcolor{blue}{T}\\
    \hline
    \textcolor{red}{F} & \textcolor{blue}{T} & \textcolor{blue}{T}\\
    \hline
    \end{array}
\end{displaymath}\\

{\large\centerline{Final Test Proposition Truth Table}}

\begin{displaymath}
    \begin{array}{|c|}
    % |c c|c| means that there are three columns in the table and
    % a vertical bar ’|’ will be printed on the left and right borders,
    % and between the second and the third columns.
    % The letter ’c’ means the value will be centered within the column,
    % letter ’l’, left-aligned, and ’r’, right-aligned.
    \hline
    (p \iff (p \lor \neg q)) \land (q \iff (p \land \neg r)) \land (r \iff (r \oplus q))\\ % Use & to separate the columns
    \hline % Put a horizontal line between the table header and the rest.
    \textcolor{red}{F}\\
    \hline
    \textcolor{red}{F}\\
    \hline
    \textcolor{blue}{T}\\
    \hline
    \textcolor{red}{F}\\
    \hline
    \textcolor{red}{F}\\
    \hline
    \textcolor{red}{F}\\
    \hline
    \textcolor{red}{F}\\
    \hline
    \textcolor{red}{F}\\
    \hline
    \end{array}
\end{displaymath} \\[0.5in]

\noindent\textbf{Conclusion}: From the final test Propositions, we can see that we only yielded one \textcolor{blue}{T}, which is in the 3rd row. Looking back at our first Truth table, this would mean: \\

{\large\centerline{ A is a knight}}
{\large\centerline{B is a knave}}
{\large\centerline{ C is a knight}}\vspace{1cm}

\noindent We can justify this conclusion by looking at C's statement and seeing that it does hold true, in which B is in fact a different type. B's statement does not hold true as C is a is knight and not a knave. A's statement about B holds true, which justifies it's statement about being a knight.\\[0.2in]

\noindent\textbf{Q2a}. Proving $\neg(\neg p \lor \neg(q \implies \neg r) \equiv p \land \neg(q \land r)$ through Truth Table\\

{\large\centerline{$(\neg(\neg p \lor \neg(q \implies \neg r))$}}

\begin{displaymath}
    \begin{array}{|c|c|c|c|c|c|}
    % |c c|c| means that there are three columns in the table and
    % a vertical bar ’|’ will be printed on the left and right borders,
    % and between the second and the third columns.
    % The letter ’c’ means the value will be centered within the column,
    % letter ’l’, left-aligned, and ’r’, right-aligned.
    \hline
    p & q & r & \neg p & \neg q & \neg r\\ % Use & to separate the columns
    \hline % Put a horizontal line between the table header and the rest.
    \textcolor{blue}{T} & \textcolor{blue}{T} & \textcolor{blue}{T} & \textcolor{red}{F} & \textcolor{red}{F} & \textcolor{red}{F}\\
    \hline
    \textcolor{blue}{T} & \textcolor{blue}{T} & \textcolor{red}{F} & \textcolor{red}{F} & \textcolor{red}{F} & \textcolor{blue}{T}\\
    \hline
    \textcolor{blue}{T} & \textcolor{red}{F} & \textcolor{blue}{T} & \textcolor{red}{F} & \textcolor{blue}{T} & \textcolor{red}{F}\\
    \hline
    \textcolor{blue}{T} & \textcolor{red}{F} & \textcolor{red}{F} & \textcolor{red}{F} & \textcolor{blue}{T} & \textcolor{blue}{T}\\
    \hline
    \textcolor{red}{F} & \textcolor{blue}{T} & \textcolor{blue}{T} & \textcolor{blue}{T} & \textcolor{red}{F} & \textcolor{red}{F}\\
    \hline
    \textcolor{red}{F} & \textcolor{blue}{T} & \textcolor{red}{F} & \textcolor{blue}{T} & \textcolor{red}{F} & \textcolor{blue}{T}\\
    \hline
    \textcolor{red}{F} & \textcolor{red}{F} & \textcolor{blue}{T} & \textcolor{blue}{T} & \textcolor{blue}{T} & \textcolor{red}{F}\\
    \hline
    \textcolor{red}{F} & \textcolor{red}{F} & \textcolor{red}{F} & \textcolor{blue}{T} & \textcolor{blue}{T} & \textcolor{blue}{T}\\
    \hline
    \end{array}
\end{displaymath} \\
\begin{displaymath}
    \begin{array}{|c|c|c|c|}
    % |c c|c| means that there are three columns in the table and
    % a vertical bar ’|’ will be printed on the left and right borders,
    % and between the second and the third columns.
    % The letter ’c’ means the value will be centered within the column,
    % letter ’l’, left-aligned, and ’r’, right-aligned.
    \hline
    q \implies \neg r & \neg(q \implies \neg r) & \neg p \lor \neg(q \implies \neg r) & \neg(\neg p \lor \neg(q \implies \neg r))\\ % Use & to separate the columns
    \hline % Put a horizontal line between the table header and the rest.
    \textcolor{red}{F} & \textcolor{blue}{T} & \textcolor{blue}{T} & \textcolor{red}{F}\\
    \hline
    \textcolor{blue}{T} & \textcolor{red}{F} & \textcolor{red}{F} & \textcolor{blue}{T}\\
    \hline
    \textcolor{blue}{T} & \textcolor{red}{F} & \textcolor{red}{F} & \textcolor{blue}{T}\\
    \hline
    \textcolor{blue}{T} & \textcolor{red}{F} & \textcolor{red}{F} & \textcolor{blue}{T}\\
    \hline
    \textcolor{red}{F} & \textcolor{blue}{T} & \textcolor{blue}{T} & \textcolor{red}{F}\\
    \hline
    \textcolor{blue}{T} & \textcolor{red}{F} & \textcolor{blue}{T} & \textcolor{red}{F}\\
    \hline
    \textcolor{blue}{T} & \textcolor{red}{F} & \textcolor{blue}{T} & \textcolor{red}{F}\\
    \hline
    \textcolor{blue}{T} & \textcolor{red}{F} & \textcolor{blue}{T} & \textcolor{red}{F}\\
    \hline
    \end{array}
\end{displaymath} \\[1in]

{\large\centerline{$p \land \neg(q \land r)$}}

\begin{displaymath}
    \begin{array}{|c|c|c|}
    % |c c|c| means that there are three columns in the table and
    % a vertical bar ’|’ will be printed on the left and right borders,
    % and between the second and the third columns.
    % The letter ’c’ means the value will be centered within the column,
    % letter ’l’, left-aligned, and ’r’, right-aligned.
    \hline
    q \land r & \neg(q \land r) & p \land \neg(q \land r) \\ % Use & to separate the columns
    \hline % Put a horizontal line between the table header and the rest.
    \textcolor{blue}{T} & \textcolor{red}{F} & \textcolor{red}{F}\\
    \hline
    \textcolor{red}{F} & \textcolor{blue}{T} & \textcolor{blue}{T}\\
    \hline
    \textcolor{red}{F} & \textcolor{blue}{T} & \textcolor{blue}{T}\\
    \hline
    \textcolor{red}{F} & \textcolor{blue}{T} & \textcolor{blue}{T}\\
    \hline
    \textcolor{blue}{T} & \textcolor{red}{F} & \textcolor{red}{F}\\
    \hline
    \textcolor{red}{F} & \textcolor{blue}{T} & \textcolor{red}{F}\\
    \hline
    \textcolor{red}{F} & \textcolor{blue}{T} & \textcolor{red}{F}\\
    \hline
    \textcolor{red}{F} & \textcolor{blue}{T} & \textcolor{red}{F}\\
    \hline
    \end{array}
\end{displaymath} \\
\begin{displaymath}
    \begin{array}{|c|c|c|}
    % |c c|c| means that there are three columns in the table and
    % a vertical bar ’|’ will be printed on the left and right borders,
    % and between the second and the third columns.
    % The letter ’c’ means the value will be centered within the column,
    % letter ’l’, left-aligned, and ’r’, right-aligned.
    \hline
    \neg(\neg p \lor \neg(q \implies \neg r)) & p \land \neg(q \land r) & \neg(\neg p \lor \neg(q \implies \neg r) \iff p \land \neg(q \land r) \\ % Use & to separate the columns
    \hline % Put a horizontal line between the table header and the rest.
    \textcolor{red}{F} & \textcolor{red}{F} & \textcolor{blue}{T}\\
    \hline
    \textcolor{blue}{T} & \textcolor{blue}{T} & \textcolor{blue}{T}\\
    \hline
    \textcolor{blue}{T} & \textcolor{blue}{T} & \textcolor{blue}{T}\\
    \hline
    \textcolor{blue}{T} & \textcolor{blue}{T} & \textcolor{blue}{T}\\
    \hline
    \textcolor{red}{F} & \textcolor{red}{F} & \textcolor{blue}{T}\\
    \hline
    \textcolor{red}{F} & \textcolor{red}{F} & \textcolor{blue}{T}\\
    \hline
    \textcolor{red}{F} & \textcolor{red}{F} & \textcolor{blue}{T}\\
    \hline
    \textcolor{red}{F} & \textcolor{red}{F} & \textcolor{blue}{T}\\
    \hline
    \end{array}
\end{displaymath} \\

{\large\centerline{Through Logical Equivalence}}
\begin{displaymath}
    \begin{array}{|c|c|c|}
    % |c c|c| means that there are three columns in the table and
    % a vertical bar ’|’ will be printed on the left and right borders,
    % and between the second and the third columns.
    % The letter ’c’ means the value will be centered within the column,
    % letter ’l’, left-aligned, and ’r’, right-aligned.
    \hline
    Steps & \neg(\neg p \lor \neg(q \implies \neg r) & Logical \: Equivalence \\ % Use & to separate the columns
    \hline % Put a horizontal line between the table header and the rest.
    1 & \neg(\neg p \lor \neg(\neg q \lor \neg r) & RBI\\
    \hline
    2 & \neg(\neg p \lor [\neg \neg q \land \neg \neg r]) & 2nd \: Demorgan's \: Law\\
    \hline
    3 & \neg(\neg p \lor (q \land r)) & Double \: Negation\: Law\\
    \hline
    4 & \neg \neg p \land \neg (q \land r)) & 2nd \: Demorgan's \: Law\\
    \hline
    5 & p \land \neg(q \land r) & Double \: Negation\: Law\\
    \hline
    \end{array}
\end{displaymath} \\

\textbf{A}: Through Truth Table and Logical Equivalence rules, we can see that

\setlength\parindent{40pt}{$\neg(\neg p \lor \neg(q \implies \neg r)$ and $(p \land \neg(q \land r)$ are both logically Equivalent} \\[2in]

\textbf{Q2b}. Proving $p \implies (p \lor q) \equiv q \lor \neg q$ through Truth Table\\

{\large\centerline{$p \implies (p \lor q)$}}

\begin{displaymath}
    \begin{array}{|c|c|c|c|c|c|c|}
    % |c c|c| means that there are three columns in the table and
    % a vertical bar ’|’ will be printed on the left and right borders,
    % and between the second and the third columns.
    % The letter ’c’ means the value will be centered within the column,
    % letter ’l’, left-aligned, and ’r’, right-aligned.
    \hline
    p & q & \neg p & \neg q & p \lor q & p \implies (p \lor q) & q \lor \neg q\\ % Use & to separate the columns
    \hline % Put a horizontal line between the table header and the rest.
    \textcolor{blue}{T} & \textcolor{blue}{T} & \textcolor{red}{F} & \textcolor{red}{F} & \textcolor{blue}{T} & \textcolor{blue}{T} & \textcolor{blue}{T}\\
    \hline
    \textcolor{blue}{T} & \textcolor{red}{F} & \textcolor{red}{F} & \textcolor{blue}{T} & \textcolor{blue}{T} & \textcolor{blue}{T} & \textcolor{blue}{T}\\
    \hline
    \textcolor{red}{F} & \textcolor{blue}{T} & \textcolor{blue}{T} & \textcolor{red}{F} & \textcolor{blue}{T} & \textcolor{blue}{T} & \textcolor{blue}{T}\\
    \hline
    \textcolor{red}{F} & \textcolor{red}{F} & \textcolor{blue}{T} & \textcolor{blue}{T} & \textcolor{red}{F} & \textcolor{blue}{T} & \textcolor{blue}{T}\\
    \hline
    \end{array}
\end{displaymath} \\

\textbf{A}: Through Truth Table, we can see that $p \implies (p \lor q)$ and $q \lor \neg q$ are both 

\setlength\parindent{40pt}{tautology, which proves that they're Logically Equivalent} \\

{\large\centerline{Through Logical Equivalence}}
\begin{displaymath}
    \begin{array}{|c|c|c|}
    % |c c|c| means that there are three columns in the table and
    % a vertical bar ’|’ will be printed on the left and right borders,
    % and between the second and the third columns.
    % The letter ’c’ means the value will be centered within the column,
    % letter ’l’, left-aligned, and ’r’, right-aligned.
    \hline
    Steps & p \implies (p \lor q) & Logical \: Equivalence \\ % Use & to separate the columns
    \hline % Put a horizontal line between the table header and the rest.
    1 & \neg p \lor (p \lor q) & RBI\\
    \hline
    2 & (\neg p \lor p) \lor q & Associative \: Law\\
    \hline
    3 & q \lor (\neg p \lor p) & Commutative\: Law\\
    \hline
    4 & q \lor T \equiv T & Domination \: Law\\
    \hline
    5 & T \equiv q \lor \neg q & Negation\: Law\\
    \hline
    \end{array}
\end{displaymath} \\

\textbf{A}: Through a chain of rules, we can see transform $p \implies (p \lor q)$ to $q \lor \neg q$, thus 

\setlength\parindent{40pt}{proving that they're Logically Equivalent}\\[0.5in]

\textbf{Q3a}. Translating each group's seating requirements into a Proposition\\

\begin{displaymath}
    \begin{array}{|c|c|}
    % |c c|c| means that there are three columns in the table and
    % a vertical bar ’|’ will be printed on the left and right borders,
    % and between the second and the third columns.
    % The letter ’c’ means the value will be centered within the column,
    % letter ’l’, left-aligned, and ’r’, right-aligned.
    \hline
    Req & Proposition\\ % Use & to separate the columns
    \hline % Put a horizontal line between the table header and the rest.
    1 & C(N,V)\\
    \hline
    2 & C(D,S)\land (C(D,F)\lor C(D,N))\\
    \hline
    3 & \neg C(F,N) \lor \neg C(F,S)\\
    \hline
    4 & C(F,D) \iff (\neg C(D,N) \land \neg C(D,F))\\
    \hline
    5 & \exists x\exists y(C(F,x) \land (F,y) \land (x \ne y) \land (F \ne x) \land (F \ne y))\\
    \hline
    6 & \forall y\exists xC(x,y)\\
    \hline
    \end{array}
\end{displaymath} \\[1in]

\noindent\textbf{Q3b}. Are any requirements satisfiable?\\

{\large\centerline{Possible Seating Arrangements}}\vspace{0.5cm}

\centerline{A: Norville - Velma - Fred - Dalphne - Scooby}
\centerline{This seating arrangement would contradict the 3rd Proposition}\vspace{0.5cm}

\centerline{B: Fred - Dalphne - Velma - Norville - Scooby}
\centerline{This seating arrangement would contradict the 2nd and 5th Proposition}\vspace{0.5cm}

\centerline{C: Scooby - Dalphne - Fred - Velma - Norville}
\centerline{This seating arrangement would contradict the 4th Proposition}\vspace{0.5cm}

\centerline{D: Dalphne - Fred - Velma - Scooby - Norville}
\centerline{This seating arrangement would contradict the 1st and 2nd Proposition}\vspace{0.5cm}

\noindent\textbf{Conclusion}: We can see that no matter what we do, we cannot satisfy the seating requirements (Why does fred have to be so picky?). Thus, the seating arrangements are NOT satisfiable.\\[0.5in]

\noindent\textbf{Q4a}. Smallest number of fake HW problems to identify the leaker?\\

\noindent We can derive this answer by using binary "logic". When we start with initially 2 CAs, we only need 1 fake homework problem to identify the leaker (since that person will leak anything). From this, if we jump to 4 CAs, we would then need 2 sets of unique fake homework problem to identify the leaker. From this logic, we can derive it as a based 2 problem/equation:\\

{\large\centerline{$2^{(n)}= 100$ , n = ?}}\vspace{0.5cm}

\noindent Where n is the number of fake homework problem that will identify the leaker out of 100 CAs. From this, we can see if n = 6, that would only cover 64 CAs, but n = 7 would cover up to 128 CAs.\\

\noindent\textbf{Conclusion:} 7 is the smallest number of fake homework problem that will identify the leaker out of 100 course assistants in the class.\\[1.5in]

\noindent\textbf{Q4b.} Max number of CAs n fake homework problem can cover?\\

\noindent\textbf{A}{: From what we derived in part a, we can see that with n being the the number of fake homework problems, 128 would be the maximum number of course assistants the class can have, while still being able to identify the leaker.\\

\noindent Alternatively, we can say that the number of CAs n fake homework problem can cover is equal to the answer of base 2.\\

{\large\centerline{$2^{(n)}= Number \; of \; CAs \; Fake \; HW \; can \; cover$}}\vspace{0.5cm}

\noindent Meaning if n = 5 for example, the maximum number of course assistants that 5 fake homework problems can still be used to identify the leaker is 32.\\[0.5in]

\noindent\textbf{Q5a.} Every inhabitant on the island tells you "Some of us are Knights and some of us are Knaves."\\

\textbf{(i) A:} That statement can be translated into:  $\exists xK(x) \land \exists x(\neg K(x))$\\

\textbf{(ii) A:} If we look at it from the perspective that everyone we talked to

\setlength\parindent{80pt}{is a Knight, then that would make the predicate statement from}

\setlength\parindent{80pt}{(i) false since if we have all Knights, some cannot be knaves and}

\setlength\parindent{80pt}{knights cannot lie. If we look at it from the perspective that}

\setlength\parindent{80pt}{everyone we talked to are Knaves, this would make the predicate}

\setlength\parindent{80pt}{false as well, as some cannot be Knights if all are Knaves. However,}

\setlength\parindent{80pt}{because knaves are liars, this means that the statement came from}

\setlength\parindent{80pt}{Knaves, as it was false. This means the inhabitants on the island are}

\setlength\parindent{80pt}{Knaves.}\\

\noindent\textbf{Q5b.} Only talked to one inhabitant and they said "All of us are Knaves."\\

\setlength\parindent{40pt}\textbf{(i) A:} That statement can be translated into:  $\forall x (\neg K(x))$\\

\setlength\parindent{40pt}\textbf{(ii) A:} Knaves are liars, so in order for the statement to be true, the inhabi-

\setlength\parindent{80pt}{tants cannot be all Knaves. There must exist at least one Knight on}

\setlength\parindent{80pt}{the island for the statement to be valid.}

\end{document}